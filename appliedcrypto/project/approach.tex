\section{Approach}

My proposed approach is to modify boto Python library to add support for AES256 local encryption. 
This addition will be an extra option that the developer can choose to use while using boto, and if client-side encryption
is invoked, it will happen transparently upon file upload and download to the users machine. Encryption will be performed
using a key specified by the user, and will never be stored in the cloud.

\subsection{Implementation Details}

boto is an open source Python library which interfaces to many Amazon Web Services API's. I will be able to fork the repository and make my changes to the full source code.
If the modifications to boto are successful, I might be able to contribute the work back into the main branch of the software.
boto supports single file uploads (up to 5 gigabytes in size) as well as a multipart upload for larger files. I will focus first on the single file upload case, and address multipart upload if there is time.

For cryptographic functions, I propose the use of the open source Pycrypto library \cite{pycrypto}. It has functions for performing DES and AES ciphers, in different modes of operation, as well as several secure hashing functions.
One thing that I will want to quantify is the cost of encrypting the data as increased run time, compared to the unencrypted file upload/download case.

