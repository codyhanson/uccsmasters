\documentclass[letterpaper]{article}
\usepackage{aaai}
\usepackage{times}
\usepackage{helvet}
\usepackage{courier}
\usepackage{listings}

\begin{document}

\title{CS5820 AI - Homework 1}
\author{Cody Hanson\\
University of Colorado Colorado Springs\\
1420 Austin Bluffs Pkwy\\
Colorado Springs, Colorado USA 80918\\
chanson@uccs.edu}
\maketitle
\begin{abstract}
\begin{quote}
This assignment's goal was to build a `rules engine' which could apply patterns to text.
As a practical application of our rules engine, we were also tasked with putting it to work using it
to build a symbolic differentiation system, which can take derivatives of arbitrary mathematical expressions.
\end{quote}
\end{abstract}

\section{Introduction}
One of the key things to consider when approaching this problem, is which
tool to use during implementation. The examples were mostly in Lisp syntax
which hinted that perhaps Lisp would be an ideal match for this kind
of problem. I initially started out using Python to code my rules engine,
but I quickly found it to be kind of unwieldy, when manipulating patterns in
the way that we needed to for this assignment.

I then switched to Lisp, and after a bit of time getting up to speed on the
language, I found that it was an ideal choice. The ability to easily parse 
and manipulate arbitrary tree structures proved invaluable for building the
rules engine and the differentiation program. Lisp was also able to
succinctly express algorithms using recursion, which help to minimize
the amount of code that I had to write.

\section{Manipulating Rules}
The core of my rules engine was implemented by 4 main functions: `sub' for
taking in a given association list and making substitutions, `match' which determined
if the left hand side of a rule matched an input string, and if it matched, generate
association list to be input into `sub', and `applyOneRule', which applied the first
matching rule from a list of rules, or none if the rules didn't match.

The way that I determined if a the left hand side of a rule matched a phrase
was to implement a `zip-uneven' function, which allowed me to match up the
elements of a phrase with the elements of a rule, to see if they were compatible.
`zip-uneve' worked by walking each list and creating pairs of items, and padding
with NIL in case one list was longer than the other. Then with the `match' function,
I would walk the pairs and determine if they were equal to each other (in the case of 
two atoms) or if one atom lined up with a `substitution pattern' such as `(? x)'.

I chose to implement `applyOneRule' with foresight to the differentiation application.
With differentiation, you woudn't want to apply more than one rule at a time, otherwise
you wouldn't get the 1st derivative, but the 2nd or 3rd.


\subsection{File input techniques}
I devised a system to be able to read rules from a file, as well as test
inputs and expected outputs. This enabled the system to be more flexible, and
easier to add new rules and test inputs, rather than hard code each test case
in the main program.

In a rules file, there was one rule per line, with the left hand side and right hand side
being separated by a space. Due to the way Lisp parsed this file, you are allowed to add a
comment to the end of a line after a valid rule, and it would be ignored.

In a test input file, there would be one test input and expected output pair per line.
I was able to compute the expected outputs by hand, because I had knowledge of the kinds of
rules that would be present. By having arbitrarily many test inputs in a file, and a
main loop in the `rules_engine_demo' program, I was able to easily add new test cases,
without changing the code, as well as automatically evaluate if the program output
matched what the expected output was.

\section{Symbolic Differentiation}

My approach to implementing a differentiation engine was to leverage the rules engine
to perform a single differentation step, based on rules stored in a file. The key
next step was to write the rules in such a way such that, if a rule required
another differentiation step, that the output of the first rule application would have
an embedded `(ddx <some expression>)' in it. After the top level differentiation step,
I input the output of that first step into a function which would recursively look
for embedded `ddx' symbols, which would require the application of another rule.
This would continue until all embedded `ddx' symbols were evaluated.


\subsection{Testing and results}
To ensure adequate coverage of all cases, I specified a test input that excercised
every differentiation rule. I also included differentiable internal functions,
to test that recursive differention was tested, such as in the rules for the cosine.

I was able to obtain successful answers for all expressions that I attempted. The
answers were programmatically checked against a human derived lisp format string
using the lisp function `tree-equal'.


\subsection{Future Work}
One way that I could improve my system is to find a way to mitigate stack overflow issues.
Due to the use of recursion, given a sufficiently complex input, it could cause the
Common Lisp interpreter to suffer a stack overflow error, due to too many recursive calls.

Partial differentiation, with respect to variables besides `x' could also be a good improvement,
to allow for use in more complex mathematical problems.

Allowing the user to have an interactive input on the command line would enable the testing of
expressions without adding them to the test input file. If possible being able to 
specify the expected output as well could allow the program to automatically check
the success of the differentiation would reduce human error, similarly to how I implemented
automatic checking in my file input method.

Another thing to consider, is that of the rules database files. Currently they are applied
in the order they are in the file, such that the first to match will win. Care should be 
taken to ensure that rules for differentiation are placed in the correct order.

Finally, given more time, it would be a big improvement to implement some simplification logic.
For instance, if there was a term such as `(difference 2 1)' this could easily be
evaluated to `1'. This would make some of the output expressions less lengthy, and 
easier for a human to process.

%adjust the style so we can display terminal output properly.
\onecolumn
%\addtolength{\oddsidemargin}{-.5in}

\newpage
\section{Appendix A - Rules Engine Code and Output}
\subsection{Rules Engine Code}
\lstinputlisting[language=lisp, frame=single]{../rules_engine.lisp}
\subsection{Rules Engine Demo Code}
\lstinputlisting[language=lisp]{../rules_engine_demo.lisp}
\subsection{Rules Engine Demo Rules}
\lstinputlisting{test_rules.txt}
\subsection{Rules Engine Demo Input}
\lstinputlisting{test_phrases.txt}
\subsection{Rules Engine Demo Output}
\lstinputlisting{rules_engine_demo_output.txt}


\newpage
\section{Appendix B - Differentiation Code and Output}
\subsection{Differentiation Code}
\lstinputlisting[language=lisp]{../derivative.lisp}
\subsection{Differentiation Rules}
\lstinputlisting{derivative_rules.txt}
\subsection{Differentiation Test Input}
\lstinputlisting{derivative_test_input.txt}
\subsection{Differentiation Test Output}
\lstinputlisting{derivative_output.txt}

\end{document}
